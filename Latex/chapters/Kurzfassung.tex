\kurzfassung

Der zunehmende Einsatz von Programmieraufgaben in der Hochschullehre stellt Lehrkräfte vor die Herausforderung, eine große Anzahl an studentischen Einreichungen effizient und gleichzeitig qualitativ hochwertig zu bewerten. Insbesondere der zeitliche Aufwand für individuelles Feedback nimmt mit wachsender Teilnehmerzahl erheblich zu. In dieser Arbeit wurde ein Verfahren zur KI-gestützten Clusterung von studentischen Programmierlösungen entwickelt, das den Bewertungsprozess entlasten und die Feedbackqualität verbessern soll.

Ziel war es, eine Pipeline zu implementieren, die Java-Dateien automatisiert einliest, in numerische Vektorrepräsentationen überführt und anschließend mittels Dimensionsreduktion und Clustering gruppiert. Durch diese Clusterungen sollen ähnliche Lösungen identifiziert werden, um repräsentative Beispiele pro Cluster für ein standardisiertes Feedback auszuwählen.

Im Verlauf der Arbeit wurden verschiedene Clustering- und Dimensionsreduktionsalgorithmen getestet und hinsichtlich ihrer Eignung bewertet. Besonderes Augenmerk lag auf der Konsistenz der Cluster bei großen und ungleichen Datensätzen sowie der Einbeziehung von Punktzahlen als ergänzendes Kriterium zur Verbesserung der Clusterqualität. Die entwickelte Lösung wurde modular umgesetzt und anhand eines beispielhaften Experiments von rund 1000 studentischen Einreichungen in Java evaluiert.

Die Ergebnisse zeigen, dass sich die entwickelte Pipeline prinzipiell eignet, Programmierlösungen automatisiert zu clustern und somit den Korrekturaufwand zu reduzieren. Gleichzeitig konnte aufgezeigt werden, dass eine rein syntaktisch-semantische Clusterung kontextabhängig nicht ausreicht, um homogene Gruppen für Feedbackzwecke zu bilden. Ergänzend wurden dafür Punktzahlenintervalle und die Konkatenation von Einreichungen eingebaut, um Clusterungen weiter einzugrenzen. Die Arbeit legt damit eine fundierte Grundlage für weiterführende Entwicklungen im Bereich automatisierter Feedbacksysteme für die Hochschullehre.

\kurzfassungEN

The increasing use of programming tasks in university teaching presents teachers with the challenge of evaluating a large number of student submissions efficiently and at the same time ensuring high quality. In particular, the time required for individual feedback increases significantly as the number of participants grows. In this thesis, a method for AI-supported clustering of student programming solutions was developed to reduce the burden on the evaluation process and improve the quality of feedback.

The goal was to implement a pipeline that automatically reads Java files, converts them into numerical vector representations, and then groups them using dimension reduction and clustering. These clusters are intended to identify similar solutions in order to select representative examples per cluster for standardized feedback.

In the course of the work, various clustering and dimension reduction algorithms were tested and evaluated for their suitability. Particular attention was paid to the consistency of the clusters in large and unequal data sets and the inclusion of scores as a supplementary criterion for improving cluster quality. The developed solution was implemented in a modular fashion and evaluated using an exemplary experiment of around 1,000 student submissions in Java.

The results show that the developed pipeline is fundamentally suitable for automatically clustering programming solutions, thereby reducing the amount of correction work required. At the same time, it was shown that purely syntactic-semantic clustering is not sufficient in context-dependent cases to form homogeneous groups for feedback purposes. To this end, score intervals and the concatenation of submissions were incorporated to further narrow down the clusters. The work thus lays a solid foundation for further developments in the field of automated feedback systems for university teaching.
