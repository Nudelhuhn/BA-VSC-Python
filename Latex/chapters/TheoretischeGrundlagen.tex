\chapter{Theoretische Grundlagen}

\section{Künstliche Intelligenz}
Künstliche Intelligenz (KI) bezeichnet die Wissenschaft und Technik der Entwicklung von Maschinen, die Aufgaben ausführen können, für die normalerweise menschliche Intelligenz erforderlich ist (Russell \& Norvig, 2021, S. 1). Der Mensch hat sich damit Systeme geschaffen, um die Kapazitäten menschlicher Intelligenz für bestimmte Aufgaben zu schonen oder zu erweitern. Dabei übernimmt die KI den Teil der Automatisierung monotoner Arbeit, also jenen Part, der durch menschliches Handeln erwiesenermaßen fehleranfällig ist, um konsistente Ergebnisse bereitzustellen. Automatisierung bedeutet in diesem Zusammenhang, dass Maschinen bestimmte Entscheidungs- und Handlungsprozesse eigenständig durchführen, ohne dass eine Person direkt in jeden einzelnen Schritt eingreifen muss. Während sich solche Systeme ursprünglich vor allem in der industriellen Fertigung etablierten, stellt sich zunehmend die Frage, wie sich vergleichbare Konzepte auf andere Bereiche übertragen lassen, wie etwa auf das Bildungswesen und hier speziell auf die Bewertung und Rückmeldung (Feedback) zu studentischen Programmierlösungen. Wie kann eine intelligente Automatisierung Lehrkräfte dabei unterstützen, qualitativ hochwertiges und individualisiertes Feedback zu generieren, ohne jede Lösung einzeln manuell prüfen zu müssen?

In dieser Arbeit wird KI anhand verschiedener Open-Source-Bibliotheken genutzt. Das System kombiniert mehrere KI-Techniken wie
\begin{itemize}
    \item Deep Learning - ein Teilbereich des Machine Learnings (ML), der künstliche neuronale Netze mit vielen Schichten verwendet, um komplexe Muster in Daten zu erkennen,
    \item Unsupervised Machine Learning - ein Verfahren, bei denen Modelle ohne beschriftete Trainingsdaten Muster oder Strukturen in den Daten erkennen, z.B. durch Clustering, und
    \item verschiedene Machine Learning Methoden, bei denen Computer aus Beispieldaten eigenständig Muster und Zusammenhänge erkennen, um daraus Vorhersagen oder Entscheidungen abzuleiten.
\end{itemize}
Das Zusammenspiel dieser Methoden führt letztlich zur Unterstützung der Lehrkräfte zur Feedbackgenerierung, was in den Ergebnissen zu sehen sein wird.

\section{Stand der Technik}
Inspiration für diese Arbeit wurde aus aktuellen Werken entnommen, wie Orvalho et al. (2022), die mit InvAASTCluster ein Verfahren zur Clusterung von Programmierlösungen mittels dynamischer Invarianten-Analyse vorstellen (vgl. \cite{Orvalho.28.06.2022}); aus Paiva et al. (2024) die AsanasCluster, ein inkrementelles k-means-basiertes Verfahren, zur Clusterung von Programmierlösungen für automatisiertes Feedback entwickelt haben (vgl. \cite{Paiva.2024}); und Tang et al. (2024) die Large Language Models\footnote{auf Textdaten trainierte KI-Modelle, die natürliche Sprache verarbeiten und generieren} (LLMs) und Clustering kombinieren, um personalisiertes Feedback in Programmierkursen zu skalieren (vgl. \cite{Tang.21.10.2024}).

Diese Quellen und weitere Recherche dienten zum Kennenlernen und zum späteren Einbinden von Algorithmen, die aufgrund ihrer besonderen Eignung zur Clusterung von Programmieraufgaben besonders herausstachen, wie 
\begin{itemize}
    \item k-means - eine Methode zur Aufteilung von N Beobachtungen in k Cluster, wobei jede Beobachtung zu dem Cluster mit dem nächstgelegenen Mittelwert gehört, der als Prototyp des Clusters dient (vgl. \cite{MacQueen.1967}), und
    \item hdbscan - eine Erweiterung des Density-Based Spatial Clustering of Applications with Noise (DBSCAN) Algorithmus, indem es eine Hierarchie von Clustern aufbaut und die stabilen Cluster über unterschiedliche Dichteebenen hinweg extrahiert (vgl. \cite{CampelloRicardoJ.G.B..2013}),
\end{itemize}
Weiterhin wurden Verfahren eingebunden die hochdimensionale Vektoren in ihren Dimensionen reduzieren, um sie für weiterführende Prozesse wie z. B. zur Clusterung und Visualisierung der Daten nutzbar zu machen. In dieser Arbeit wurden
\begin{itemize}
    \item Principal component analysis (PCA) - ein Verfahren zur Projektion hochdimensionaler Daten in einen lineareren Unterraum mit geringerer Dimension, indem neue Achsen, entlang derer die Daten am stärksten streuen, berechnet werden und stellt die Daten entlang dieser Achsen dar (vgl. \cite{KarlPearson.1901}),
    \item t-distributed stochastic neighbor embedding (t-SNE) - ein Verfahren zur Visualisierung hochdimensionaler Daten, indem es berechnet wie ähnlich Punkte zu ihren Originaldaten sind, um sie entsprechend weit auseinander oder nahe zusammen zu platzieren (vgl. \cite{LaurensvanderMaatenundGeoffreyHinton.2008}), und
    \item Uniform Manifold Approximation and Projection (UMAP) - ein Verfahren welches Topologie und Geometrie nutzt, um skalierbare, strukturtreue EIinbettungen in niedrigere Dimensionen zu erreichen (vgl. \cite{McInnes.09.02.2018}).
\end{itemize}

% Methodik
% Aufgrund der weltweiten Etablierung und der Bereitstellung von Bibliotheken die die obigen KI-Techniken bereitstellen, wurde für die Programmierung des Systems Python als Programmiersprache benutzt. Darüber hinaus zeichnet sich Python durch eine einfache, lesbare Syntax und eine aktive Entwickler- und Forschungsgemeinschaft aus, was die Entwicklung, Erweiterung und Wartung des Systems erleichtert.