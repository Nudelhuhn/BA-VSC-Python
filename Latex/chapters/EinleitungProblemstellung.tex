\chapter{Einleitung und Problemstellung}

% Begonnen werden soll mit einer Einleitung zum Thema, also Hintergrund und Ziel erläutert werden.

% Weiterhin wird das vorliegende Problem diskutiert: Was ist zu lösen, warum ist es wichtig, dass man dieses Problem löst und welche Lösungsansätze gibt es bereits. Der Bezug auf vorhandene oder eben bisher fehlende Lösungen begründet auch die Intention und Bedeutung dieser Arbeit. Dies können allgemeine Gesichtspunkte sein: Man liefert einen Beitrag für ein generelles Problem oder man hat eine spezielle Systemumgebung oder ein spezielles Produkt (z.B. in einem Unternehmen), woraus sich dieses noch zu lösende Problem ergibt.

% Im weiteren Verlauf wird die Problemstellung konkret dargestellt: Was ist spezifisch zu lösen? Welche Randbedingungen sind gegeben und was ist die Zielsetzung? Letztere soll das beschreiben, was man mit dieser Arbeit (mindestens) erreichen möchte.

Für Lehrkräfte an Hochschulen oder Universitäten kann das Kontrollieren und Bewerten von studentischen Einreichungen je nach Anzahl zu einer großen Herausforderung werden. Gerade bei einer hohen Anzahl an Abgaben steigt der Korrekturaufwand erheblich, was den zeitlichen Rahmen für individuelles Feedback einschränken könnte. Eine Untersuchung zeigte, dass das Kontrollieren und Bewerten von Arbeiten der Hauptfaktor für Arbeitsbelastung und Beeinträchtigung des Wohlbefindens ist (vgl. \cite{Jerrim.2021}). In der Informatik könnte es sich hier auf Programmieraufgaben beziehen. Dabei müssen Lehrkräfte konsistente Bewertungen abliefern, während jede Abgabe unterschiedliche Syntax und Semantik beinhalten könnte.

Eine Lösung dieses Problems bieten etablierte Systeme zur automatischen Auswertung von Programmieraufgaben. Systematische Übersichtsarbeiten zeigen, dass viele Werkzeuge vorwiegend auf Unit-Tests oder statische Analysen setzen, was meist zu eher generischem Feedback führt (vgl. \cite{Messer.2023}). Die Hoschule Trier benutzt beispielsweise ASB - Automatische Software-Bewertung\footnote{https://www.hochschule-trier.de/informatik/forschung/projekte/asb}. Nachdem hierbei Studierende die von der Lehrkraft gestellte Aufgabe bearbeitet haben, können sie online ihre Lösungen hochladen. Das Programm prüft danach nach statischen Kriterien, ob z. B. alle benötigen Dateien hochgeladen wurden, ob sie der Namenskonvention entsprechen, etc. Daraufhin wird das hochgeladene Programm mit Testdaten ausgeführt und geprüft, ob die zu erwarteten Ergebnisse ausgegeben werden. Sollte das nicht der Fall sein, wird eine Fehlermeldung ausgegeben, dass das Programm oder bestimmte Module nicht erwartungsgemäß funktionieren. 

Das erzeugte Feedback solcher statischen Systeme dient zur Orierentierung, jedoch weniger zur Fehlersuche, da es recht allgemein gehalten ist. Um die Feedbackgenerierung zu verbessern könnten KI-gestütze Verfahren eingesetzt werden. Damit befassten sich beispielsweise die Autoren der wissenschaftlichen Arbeiten \dots (Hier Quellen einfügen und erläutern).

Dazu wurde in dieser Arbeit versucht, einen Schritt vor der Feedbackgenerierung zu entwickeln. Er befasst sich mit der KI-gestützten Clusterung studentischer Programmierlösungen. Dieser Ansatz ermöglicht es für mehrere Einreichungen ein gemeinsames Feedback zu generieren, indem sie nach Ähnlichkeit in Bezug auf Syntax und Semantik geclustert bzw. gruppiert werden. Weiterführende Prorgamme oder auch Lehrkräfte könnten dann einen Kandidat pro Cluster wählen, Feedback erzeugen und dieses an alle anderen Teilnehmer des Clusters weiterleiten. Dies könnte eine erhebliche Zeitersparnis zur Folge haben und weiterhin mehr Spielraum für präziseres individuelles Feedback ermöglichen. 